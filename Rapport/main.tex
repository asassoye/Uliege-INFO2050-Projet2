\input{preamble.tex}
\usepackage{clrscode3e}

\begin{document}

\subfile{titlepage.tex}

\section{Analyse théorique}
\subsection{} %a
	Pour implémenter la structure Union-Find à l'aide d'un arbre binaire, nous avons décidé de créer un vecteur dont chaque élément est initialement la racine d'un arbre binaire. Chaque arbre du vecteur représente donc un singleton. Lorsque nous procédons a une union, nous utilisons l'implémentation des arbres binaires vue au cours théorique.

\subsection{} %b
	\begin{tabular}{|l||c|c|}
	\hline
  & Liste & Arbre\\
  \hline\hline
  Fonction UfUnion & & O(h)\footnote{h étant le nombre d'étage qu'il faut monter pour arriver à la racine} \\
  Fonction UfFind & &
  \\
  \hline
\end{tabular}

\subsection{La structure de labyrinthe} %c
La structure maze est composée de 4 éléments.
\begin{itemize}
\item size : contient la taille du labyrinthe. Pour une taille N, le labyrinthe sera un carré N x N.
\item unionFind : contient l'ensemble disjoint qui a été utilisé pour créer le labyrinthe.
\item neighbours : est un vecteur contenant les paires de cellules étant voisines et un booléen égal à \textit{true} si elles sont séparées par un mur, à \textit{false} sinon.
\item convert : est une matrice permettant de retrouver le numéro d'une case du labyrinthe à partir de ses coordonnées.
\bigbreak
En effet, nous avons choisi d'utiliser la plupart du temps des entiers pour identifier les cases afin de pouvoir utiliser la structure \textit{UnionFind} préalablement définie (et qui contenait des entiers). Ainsi nous représentons chaque case par un chiffre de $0$ à $N *N - 1$
\end{itemize}

\subsection{Pseudocode} %d
\begin{codebox}
\Procname{$\proc{mzCreate}(size)$}
\li $\proc{Maze } maze$
\li $maze.size = size$
\li $maze.neighbours =$ all pairs of cells that are neighbours with a wall between each of them
\li $maze.unionFind = \proc{ufCreate}(size)$
\li \While $\proc{ufComponentsCount}(maze.unionFind) > 1$
\Do
\li 	$(coord1, coord2) =$ a random pair of neighbours
\li		\If There is a wall
\li \Then $\proc{mzSetWall}(maze, coord1, coord2, false)$ //Remove the wall
\li	$\proc{ufUnion}(maze.unionFind, coord1, coord2)$
\End
\End
\li \Return $maze$
\End
\end{codebox}

\begin{codebox}
\Procname{$\proc{mzIsValid}(maze)$}
\li \If $\proc{ufComponentsCount}(maze.unionFind) > 1$
\li \Then \Return $false$
\li \Else
\li \Return $true$
\End
\End
\end{codebox}
	
\subsection{Complexité en temps avec UnionFindList} %e
Pour connaitre la complexité de \proc{mzCreate} il faut d'abord

\proc{mzIsValid} est constant puisqu'il s'agit simplement d'aller lire la taille de l'\textit{UnionFind}.

\section{Analyse empirique}
\subsection{} %a
	\includegraphics{Tests/List/list}
	\includegraphics{Tests/Tree/tree}
	

\end{document}
